\documentclass[11pt,a4paper]{moderncv}

\moderncvtheme[blue]{classic}
\usepackage[utf8x]{inputenc}
\usepackage[scale=0.8]{geometry}
\usepackage[russian]{babel}

\usepackage[unicode]{hyperref}
\definecolor{linkcolour}{rgb}{0,0.2,0.6}
\hypersetup{colorlinks,breaklinks,urlcolor=linkcolour, linkcolor=linkcolour}

\firstname{Артем}
\familyname{Смирнов}
\address{}{Муром, Россия}
\mobile{+7(920)944-33-40}
\email{artyom\_smirnov@icloud.com}
\homepage{http://artyom-smirnov.info}

\makeatletter
\renewcommand*{\bibliographyitemlabel}{\@biblabel{\arabic{enumiv}}}
\makeatother

\renewcommand{\rmdefault}{cmr}
\renewcommand{\sfdefault}{cmss}
\renewcommand{\ttdefault}{cmtt}

\begin{document}
\maketitle

\section{Интересы}
\cvline{}{
  Мое основное направление работы --- системное программирование: разработка системного ПО и СУБД.\newline{}
  Основной язык, который я использую --- это C++.\newline{}
  Я люблю открытое ПО и утилиты, которые я применяю и в профессии и в хобби.}
  
\section{Образование}
\cventry
  {Сентябрь 2004 - Июль 2009}
  {Информационные системы}
  {Муромский Институт (отделение) Владимирского Государственного Института, Муром, Россия}
  {}{}
  {Инженер}

\section{Опыт работы}
\cventry
  {Октябрь 2013 - настоящее время}
  {Разработчик}
  {Корпорация Ред Софт}
  {}{}
  {
    Разработка и поддержка Red Database
  }
\cventry
  {Декабрь 2009 - Сентябрь 2013}
  {Разработчик}
  {SciDB}
  {}{}
  {Парсинг, трансляция и обработка запросов\newline{}
   Клиентские утилиты\newline{}
   Клиентская библиотека\newline{}
   Java/JDBC драйвер\newline{}
   Система сборки и пакетирования}
\cventry
  {Май 2007 - Август 2009}
  {Разработчик}
  {Корпорация Ред Софт}
  {}{}
  {
    Улучшения безопасности в Red Database - форке Firebird RDBMS\newline{}
    Механизм контроля целостности метаданных в Red Database\newline{}
    Портирование MTDORB с Kylix на Free Pascal\newline{}
    Портирование и поддержка на Linux утилиты репликации данных Firebird/Red Database
  }

\section{Навыки разработки}
\subsection{Языки программирования}
\cvline{}{C++, C, Python, Bash, Java, LaTeX, Pascal}{}

\subsection{Утилиты}
\cvline{}{gcc, gdb, valgrind, gprof, oprofile, gcov, lcov, bison, flex, cmake, git, svn, ant, maven, qmtest}

\section{Другие навыки}
\cvcomputer{IDE}{Eclipse, QtCreator, Xcode}{}{}
\cvcomputer{OS}{GNU Linux, OS X}{}{}

\section{Проекты}
\subsection{C++}
\cvline{Red Database}{
  \url{http://www.red-soft.biz/en/reddatabase_product.html}\newline{}
  Форк Firebird DBMS с улучшенной безопасностью для сертификации ФСТЭК
}

\cvline{SciDB}{
  \url{http://www.scidb.org}\newline{}
  Распределенная база данных для научных экспериментов с моделью данных основанной на массивах
}

\subsection{Python}
\cvline{scidb4py}{
  \url{https://github.com/artyom-smirnov/scidb4py}\newline{}
  Реализация драйвера для SciDB на чистом Python
}

\subsection{Pascal}
\cvline{MTDORB}{
  \url{http://sourceforge.net/projects/mtdorb/}\newline{}
  Открытый ORB для Pascal
}

\section{Знание иностранных языков}
\cvline{Английский язык}{Читаю, могу объясняться}

\section{Публикации}

\cvline{}{Контроль доступа к системному каталогу в СУБД <<Ред <<База Данных>>\newline{}
Ж-л <<Информационная безопасность регионов>>, Саратов, 2009, Смирнов А.В., Симаков Р.А.}

\cvline{}{Архитектура сетевого взаимодействия СУБД SciDB\newline{}
Системы и методы обработки и анализа данных: Сборник статей молодых исследователей, Выпуск 1, 2009, Смирнов А.В.}

\cvline{}{Обзор модели данных СУБД SciDB\newline{}
Системы и методы обработки и анализа данных: Сборник статей молодых исследователей, Выпуск 1, 2009, Смирнов А.В.}

\cvline{}{Особенности распределения данных в многомерной СУБД SciDB\newline{}
Алгоритмы, методы и системы обработки данных: cборник научных статей, Выпуск 14, 2009, Смирнов А.В.}

\cvline{}{Overview of SciDB: Large Scale Array Storage, Processing and Analysis\newline{}
J. Rogers, R. Simakov, E. Soroush, P. Velikhov, M. Balazinska, D. DeWitt, B. Heath, D. Maier, S. Madden, J. Patel, M. Stonebraker, S. Zdonik, A. Smirnov, K. Knizhnik, Paul G. Brown SIGMOD 2010}

\cvline{}{Разработка модели оптимизатора запросов\newline{}
Алгоритмы, методы и системы обработки данных: cборник научных статей, Выпуск 15, 2010, Смирнов А.В., Пугин Е.В.}

\cvline{}{Подготовка запросов к оптимизации и выполнению в СУБД SciDB\newline{}
Всероссийские научные <<Зворыкинские чтения – 2010>>, г. Муром., 2010, Смирнов А.В.}

\cvline{}{Оптимизация многомерных распределенных запросов в СУБД SciDB\newline{}
Всероссийские научные <<Зворыкинские чтения – 2010>>, г. Муром., 2010, Смирнов А.В.}

\cvline{}{Распределенный подход к поиску в пространстве оптимизатора запросов\newline{}
Алгоритмы, методы и системы обработки данных: cборник научных статей, Выпуск 18, 2011, Смирнов А.В.}

\cvline{}{Проблема оптимизации запросов в расширяемой СУДБ SciDB\newline{}
Алгоритмы, методы и системы обработки данных: cборник научных статей; Выпуск 18, 2011, Смирнов А.В.}

\cvline{}{Динамически расширяемый оптимизатор запросов\newline{}
Международная конференция <<Проблемы информатики и моделирования (ПИМ-2011)>>, 2011, Смирнов А.В., Пугин Е.В.}

\cvline{}{Целеориентированный распределенный подход к оптимизации запросов в СУБД\newline{}
V Международная научно-практическая конференция <<Научное творчество XXI века, 2012, Смирнов А.В.}

\section{Ссылки}
\cvline{Персональная страница}{\url{http://www.artyom-smirnov.info/}}
\cvline{LinkedIn}{\url{http://www.linkedin.com/in/artyomsmirnov}}
\cvline{GitHub}{\url{https://github.com/artyom-smirnov}}

\end{document}
